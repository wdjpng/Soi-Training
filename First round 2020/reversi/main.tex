\title{Eine Lösung für das Problem Reversi, Version 1.0}
\author{
	wdjpng / Lukas Münzel
}
\date{05. Oktober 2019}

\documentclass[12pt]{article}


\usepackage[ngerman]{babel}
\usepackage{titlesec}
\usepackage{filecontents}
\usepackage{hyperref}
 \usepackage{algorithm}
\usepackage{algorithmic}
\usepackage[bottom]{footmisc}
\begin{document}
	\maketitle
\newpage
	\section{Wie funktioniert mein Algorithmus?}
	Es werden zwei Indexe, $i_s$ und $i_e$, definiert .  Nun wird $i_s$ auf 0 und $i_e$ auf die Länge des Inputarrays weniger eins gesetzt. Die Summe $s$ der minimal benötigten Operationen $s$ wird auf null gesetzt. Sei zudem $n$ die Länge des Inputarrays, $n_l$ die Anzahl Steine, welche links hinzugefügt wurden und $n_r$ die Anzahl Steine, welche rechts hinzugefügt wurden. Zu Beginn ist $n_r = n_l = 0$.
	\\\\
	1. $i_s$  wird um eins erhöht. Wenn nun $input[i_s - 1] \neq input[i_s]$, wird $s$ um $i_s + s + n_l$ und dann $n_l$ um eins erhöht.  Dies entspricht dem hinzulegen eines Steines an das linke Ende. Wenn nach diesen Schritten $i_s = i_e$ ist, wird $s$ als Lösung zurückgegeben\\\\
	2. $i_e$  wird um eins niedriger gemacht. Wenn nun $input[i_e + 1] \neq input[i_e]$, wird $s$ um $n - i_e + n_r$ und dann $n_r$ um eins erhöht. Dies entspricht dem hinzulegen eines Steines an das rechte Ende. Wenn nach diesen Schritten $i_s = i_e$ ist, wird $s$ als Lösung zurückgegeben.\\\\
	3. Schritte 1. und 2. werden so lange wiederhohlt, bis $s$ ausgegeben wird.  Wenn in der letzten Ausführung der Schritte $input[i_s - 1] \neq input[i_s]$ war, wird 1. in der nächsten Ausführung der Schritte nicht ausgeführt. Umgekehrt wird, wenn $input[i_e + 1] \neq input[i_e]$ war, wird 2. in der nächsten Ausführung der beiden Schritte nicht ausgeführt\\\\
	
	\section{Argumentation, weshalb der Algorithmus korrekt ist}
	Für jede Veränderung in der Farbe des Inputs muss ein Stein auf eine Seite hinzugefügt werden, da pro hinzugefügtem Stein  ein Stein mehr die selbe Farbe wie andere Steine bekommt. Nun ist die Frage, weshalb es immer optimal ist, den Stein auf die Seite zu legen, auf dem die Distanz zum ersten Stein einer anderen Farbe minimal ist. 
	\subsection{Wird wirklich immer die aktuell mindeste Distanz zum Stein einer anderen Farbe genommen} Beide Indexe werden nacheinander um eins vom jeweiligen Ende weggeschoben, wenn ein Stein auf ihre Seite hinzugefügt wird, bleiben sie für einen Schritt am selben Ort des ursprünglichen Arrays, nur der Index Pointer schreitet also um eins von seinem Ende weg, womit der Abstand zum jeweiligen Ende immer minimal ist. 
	
\subsection{Weshalb führt das wiederhohlte wählen der kleinsten Distanz zur optimalen Lösung} Angenommen, dass die optimale Lösung mindestens ein Mal nicht den Stein auf die Seite, legen würden, bei der die Distanz vom Ende zum sich zu verändernden Stein minimal ist.\\\\ Demnach wäre ihre minimale Kostensumme $s_o$ kleiner als das $s$ meiner Lösung wäre. Für das Umdrehen von mindestens einem Stein hat diese aber  $s_0$ um mindestens $k$ mehr erhöht als $s$ (da eine nicht-kürzeste Distanz gewählt wurde). Dadurch können aber höchstens $k$ Kosten eingespart werden, da nach dem $k$ Steine auf die andere Seite gelegt wurden, also $k$ kosten gespart wurden, die Distanz zum ursprünglichen Input auf beiden Seiten gleich ist. Allerdings gilt $s_o + k - k = s_o$,  was heisst dass man durch Wählen einer nicht kürzersten Distanz keine kleineren Kosten als durch das wiederhohlte Wählen einer kürzesten Distanz bekommt. Demnach ist $s_o = s$, womit meine Lösung die optimale Lösung ist.
	

	\begin{algorithm}
		\caption{Pseudocode für Reversi Subtask 4}
		\begin{algorithmic}
			\STATE{startIndex = addedDiscsAtTheStart = addedDiscsToTheEnd = 0}\\
			\STATE{startFlag = endFlag = false}\\
			\STATE{endIndex = n-1}\\\
			\WHILE{$true$}
	
	\IF{$!startFlag$ 	\COMMENT{Wenn in der letzten Runde kein Stein links dazugelegt wurde}}
	\STATE{startIndex++}
	\IF{$input[startIndex] \neq lastStartChar$  \COMMENT{Soll links ein Stein dazugelegt werden}}
	\STATE{operationCounter+= start + 1 + addedDiscsAtTheStart}
	\STATE{addedDiscsAtTheStart++}
	\STATE{startFlag = true}
	\ENDIF
	\ELSE \STATE{startFlag=false}
	\ENDIF 
		\vspace{\baselineskip}% Insert a blank line
	\IF{$startIndex == endIndex$\COMMENT{Wenn sich die beiden Pointer getroffen haben, also alle Steine die selbe Farbe haben}}
	\STATE{break}
	\ENDIF
		\vspace{\baselineskip}% Insert a blank line
	\IF{$!endFlag$ \COMMENT{Wenn in der letzten Runde kein Stein rechts dazugelegt wurde}}
		\STATE{endIndex++}
		\IF{$input[endIndex] \neq lastEndChar$ \COMMENT{Soll rechts
				
				 ein Stein dazugelegt werden}}
		\STATE{operationCounter+= input.size() - endIndex + addedDiscsAtTheEnd}
		\STATE{addedDiscsAtTheEnd++}
		\STATE{endFlag = true}
		\ENDIF
		\ELSE \STATE{endFlag=false}
	\ENDIF
		\vspace{\baselineskip}% Insert a blank line
		\IF{$startIndex == endIndex$ \COMMENT{Wenn sich die beiden Pointer getroffen haben, also alle Steine die selbe Farbe haben}}
		\STATE{break}
		\ENDIF
		\vspace{\baselineskip}% Insert a blank line
		\STATE{lastStartChar = input[startIndex]}
		\STATE{lastEndChar = input[endIndex]}
	\ENDWHILE
		\end{algorithmic}
	\end{algorithm}
	
	\section{Analyse zu Laufzeit und Speichernutzung}
	\subsection{Laufzeit}
	Der Algorithmus führt pro Element im Array eine nicht mit der Grösse des Arrays wachsende Anzahl Operationen aus. Die Laufzeit dieses Algorithmus ist folgich $O(n)$
	\subsection{Speichernutzung}
	Da keine Arrays, Vektoren oder derartiges verwendet werden, die Anzahl an benötigten Variablen nicht mit der Grösse der Eingabe steigt und auch keine Elemente der Eingabe verändert werden müssen, wird $O(1)$ Speicher benutzt.

	
\end{document}